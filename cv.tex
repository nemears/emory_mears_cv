\documentclass[10pt]{article}
\usepackage[margin=0.5in]{geometry}
\usepackage{hyperref}
\begin{document}

%  TITLE
\paragraph{\centering \huge Nicholas Mears \\ }
\paragraph{\centering nickmears2@gmail.com (203)-913-9891 \url{https://github.com/nemears}\\}

% preamble
An imaginative software engineer with determination to deliver well thought out honest work that is modern and fast. Well rounded with experience, determined to provide insight to create simple and functional code to get the job done.

% work experience
\section*{Work Experience}
\begin{tabular}[t]{p{4cm} p{14cm}}
  \hline \\
    \raggedright
    \large \textbf{Software Systems Engineer} &
    \textit{MITRE, 202 Burlington Rd, Bedford MA \hfill October 2019 - Current} \newline
	  Intermediate level software engineer for the Emerging Systems Engineering Technologies department at MITRE. Primary responsibility of the job was to support sponsor projects in the role of connecting different digitial tools together to expand their use. The work focused on modeling and simulating software using SysML and UML modeling languages in tandem with simulation tools, real world api's and hardware.
\end{tabular}

% projects
\section*{Projects}
\begin{tabular}[t]{p{4cm} p{14cm}}
  \hline
  \multicolumn{2}{c}{} \\
  \large \textbf{uml-cpp} &
  \textit{Personal project \hfill June 2020 - current} \newline
  Working with antiquated apis and not being able to directly interact with those apis from embedded systems I decided to take on the responsibility of creating a modular api in c++ that follows the uml 2.5 specification in my free time. The project helped refine my modern c++ programming skills. Source code: \url{https://github.com/nemears/uml-cpp} \\
  \multicolumn{2}{c}{} \\
  \large \textbf{Physics Tool UML Integration} &
  \textit{Project: UML plugin for simulation, Mitre \hfill November 2020 - October 2021} \newline
  Lead developer for a tool that linked the syntax of physics simulation engine to a UML diagramming tool, Magicdraw. The physics simulation tool had its own syntax and filetype that had not been parsed into and emit out from UML diagrams before. The task took the engines syntax and mapped it to visual diagrams within the UML tool and used those diagrams edited or not in the tool to export back out to functional code. \\
  \multicolumn{2}{c}{} \\
  \raggedright
  \large \textbf{Simulation Tool for Aircraft System Testbed} &
  \textit{Project: Aircraft testbed, Mitre \hfill January 2021 - current} \newline
  Lead developer for reusing an aircraft simulation tool as a pipeline to test software against the physical inputs from the overall system. The work involved taking the events from the simulation and sending messages over a message bus as input to the systems to test. In addition worked on porting windows c to Posix to have simulation engine containerized and easily deployable.\\
  \multicolumn{2}{c}{} \\
  \raggedright
  \large \textbf{Publication on Non-Equilibrium Thermodynamics} &
  \textit{Worcester Polytechnic Institute \hfill June 2018 - May 2019} \newline
  WPI senior year MQP capstone, I worked in a lab as part of a team where I developed python convolutional video analysis algorithms. My work was included in a publication in the journal Nature: \url{https://arxiv.org/pdf/1812.06002.pdf} \\
\end{tabular}

% education
\section*{Education}
\begin{tabular}{l l}
  \hline
  \multicolumn{2}{c}{} \\
  \begin{minipage}[t]{7cm}
    \begin{flushleft}
      \large \textbf{Worcester Polytechnic Institute:}
    \end{flushleft}
  \end{minipage} & 
  \begin{minipage}[t]{11cm}
    \begin{flushleft}
      Bachelors of Science in Physics with a minor in Computer Science \\ 
      GPA: 3.75 Graduated May 2019 with High Distinction
    \end{flushleft}
  \end{minipage}
\end{tabular}

% skills
\section*{Skills}
\begin{tabular}{p{18.5cm}}
  \hline
  \multicolumn{1}{c}{} \\
  C++, java, python, POSIX, bash, concurrent programming, socket programming, source control, jira, linear algebra, differential equations, graph theory, embedded systems, UML, modeling and simulation
\end{tabular}

\end{document}
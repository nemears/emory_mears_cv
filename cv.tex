\documentclass[11pt]{article}
\usepackage[margin=0.5in]{geometry}
\usepackage{hyperref}
\usepackage{enumitem}
\usepackage{xcolor}
\hypersetup{
  colorlinks=true,
  linkcolor=red,
  linkbordercolor=red,
}
\begin{document}

%  TITLE
\paragraph{\centerline{\huge Emory Mears}}
\paragraph{\centering she/her $\star$ emorymears@gmail.com $\star$ (203)-913-9891 $\star$ \url{https://github.com/nemears}\\}

% preamble
An innovative software engineer with a passion for computers, and modern simple programming. She is well rounded with experience,
determined to provide insight to create efficient and functional code, and learn some new concepts on the way there. 

\section*{Skills}
\begin{tabular}{p{18.5cm}}
  \hline
  \multicolumn{1}{c}{} \\
  Low level fast programs (C++, C, Rust), Full Stack Programming (JavaScript, Java, Python, TypeScript), 
  Linux System Administration (Bash, DHCP, Systemd, Nix), Presentation (Powerpoint, Video Editing)
\end{tabular}

% experience
\section*{Experience}
\begin{tabular}{p{18.5cm}}
    \hline
    \multicolumn{1}{c}{} \\

     % uml.cafe
     \large \textbf{\href{https://uml.cafe}{uml.cafe} Modeling Platform} \normalsize \textit{Lead developer, \hfill Nov 2023 - Current}

     \> Uml Cafe is a live, multi-user, UML diagramming platform hosted on the web whose development was led 
     by Emory. She developed the protocol and API's used to interact with a UML project, and built the platform up from there. She wrote the
     entire Rust and C++ backend, and lead a small team of two in creating the frontend client. The platform is hosted live on 
     \url{https://uml.cafe} with a concise demo video on the welcome screen. She maintains the server and provides regular updates.\\
 
     \multicolumn{1}{c}{} \\

    %MITRE
    \large \textbf{Software Systems Engineer} \normalsize \textit{MITRE, 202 Burlington Rd, Bedford MA \hfill October 2019 - Nov 2023}

    \> Software Systems Engineer for the Emerging Systems Engineering Technologies department at MITRE. The following is the 
    description of the projects and tasks that she completed, most recent to least:\\ 

    \begin{itemize}[noitemsep,topsep=0pt]
      % \setlength\itemsep{0em}
      \item A C++ model of a system to be used for testing, mostly working on the routing of data throughout it so it could 
      integrate well with different open protocols and standards.
      \item A set of tools used for identifying important connections in graph models in MagicDraw of computer networks populated from SQL 
      databases and Kafka brokers.
      \item Integrating a simulation environment with the modeling tool MagicDraw, this involved 
      a set of static analysis tools that looked at the Abstract Syntax Tree of the simulation's code and populating diagrams and 
      tables within the modeling software.
      \item A tool using proxmox the virtualization manager to spin up qemu virtual machines and 
      set up a network based on configuration files supplied. The program would then run MITRE's caldera ("red team" hacker simulator) on 
      the virtualized network and machines.
    \end{itemize}\\

    % MQP
    \large \textbf{Thermodynamics Research} \normalsize \textit{WPI, 100 Institute rd, Worcester MA \hfill June 2018 - June 2019}

    \> While working on the WPI senior year capstone project she was able to work in a lab and assisted with some image analysis 
    algorithms written in python that were used and eventually cited in a \href{https://arxiv.org/pdf/1812.06002.pdf}{publication in Nature}
    to validate some non-equilibrium thermodynamics theory a staff member was working on.

\end{tabular}

% Education
\section*{Education}
\begin{tabular}{l l}
  \hline
  \multicolumn{2}{c}{} \\
  \begin{minipage}[t]{7cm}
    \begin{flushleft}
      \large \textbf{Worcester Polytechnic Institute:}
    \end{flushleft}
  \end{minipage} & 
  \begin{minipage}[t]{11cm}
    \begin{flushleft}
      Bachelors of Science in Physics with a minor in Computer Science \\ 
      GPA: 3.75 Graduated May 2019 with High Distinction
    \end{flushleft}
  \end{minipage}
\end{tabular}

\end{document}